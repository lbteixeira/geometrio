\documentclass[twoside,12pt, a4paper]{memoir}
\usepackage[utf8]{inputenc}
\setlength{\parindent}{0em}
\setlength{\parskip}{1em}
\usepackage{algorithm}
\usepackage{amsmath}
\usepackage[noend]{algpseudocode}
\usepackage{enumitem}
\usepackage{hyperref}
\hypersetup{
    colorlinks=true,
    linkcolor=blue,
    citecolor=red,
}
\algnewcommand\And{\textbf{and}}
%\usepackage[scaled]{helvet}
%\renewcommand\familydefault{\sfdefault} 
%\usepackage[T1]{fontenc}

\usepackage[top=3cm, bottom=3cm, left=2.5cm, right=2.5cm, bindingoffset=1cm]{geometry}

\title{Notes on Computational Geometry Algorithms}
\author{Lucas Teixeira}
\date{\today}

\begin{document}
\setcounter{secnumdepth}{3}
\maketitle
\tableofcontents

\chapter{Preface}\label{sec:intro}

Imagine yourself running all excited to your best friend to say that, after two weeks of intense work, you finally managed to implement an algorithm to check if a point lies inside or outside a polygon (we are going to learn this in Section~\ref{sec:point}). He could easily answer you: ``Yeah$\ldots$ right$\ldots$ but I guess even a small child could tell you that very easily. Actually, if you give me two weeks, I can even train a chimpanzee to do that!''. Perhaps your friends aren't as mean as mine, but you got the point.

\chapter{Point inclusion in a polygon}\label{sec:point}

The first problem we want to solve is:

Given a point $q$ and a polygon $P$, determine if $q$ is inside or outside $P$.

Here are some important definitions and conventions:

\begin{enumerate}
  \item The vertices of $P$ will be organized in counterclockwise direction, i.e., the interior of the polygon is at the left of the line determined by two consecutive vertices Figure~\ref{}.
  \item A polygon $P$ is convex if along with any two points $a, b \in P$, it contains the entire segment $ab$ (Figure~\ref{}). None of its vertices exceeds $\pi$.
  \item A polygon $P$ is non-convex if for any two points $a, b \in P$, it doesn't contain the entire segment $ab$ (Figure~\ref{}). It necessarily have at least one vertex exceeding $\pi$.
\end{enumerate}

\section{Basic algorithm}\label{subsec:point-basic}

One of the most simple algorithms that could be used to solve this problem is:

\begin{algorithm}
\caption{$PointInclusion(q, P)$}
\textit{Input.} A point $q$ and a polygon $P$. \\
\textit{Output.} If the point is inside the polygon or not (boolean).
\begin{algorithmic}[1]
    \State{Draw a horizontal line $l_{0r}$ from $q$ to its left}
    \State{Move from infinity to $q$, following $l_{0r}$, and count the number $N$ of intersections of $l_{0r}$ with $P$}
    \If{$N$ is odd}
        \State{$q$ is inside $P$}
    \Else{}
        \State{$q$ is outside $P$}
    \EndIf{}
\end{algorithmic}\label{inclusion-naive}
\end{algorithm}

Algorithm~\ref{inclusion-naive} is extremely simple, and any person could follow it without any problems. Nevertheless, when it comes to its computational implementation, some questions arise:

\begin{itemize}
  \item How exactly one can draw a line from a point to its left?
  \item How to move from the infinity to a point?
  \item More importantly, how to check if two lines ($l_{0r}$ and a vertex of $P$) intersect each other?
\end{itemize}

Here is where we need to start to \textit{think computationally}. First, we don't \textit{actually} need to draw a line from $q$ to its left. Since we know we are working with a horizontal line, we only need to know the coordinates of point $q$, and that's it.

Second, we are not going to move from the infinity to the point $q$, this is only a figure of speech, let's say. What we are actually going to do will be clear in a minute.

The third question is more interesting. The conditions to say if a horizontal line passing through $q$ intersects an arbitrary line $l_{ab}$ are:

\begin{enumerate}
  \item The lower point of $l_{ab}$ lies below $q$.
  \item The upper point of $l_{ab}$ lies above $q$.
  \item $q$ lies to the right from the line $\overrightarrow{l}_{ab}$, that goes from $a$ to $b$.
\end{enumerate}

But this doesn't really help, since we have now two \textit{extra} questions:

\begin{enumerate}
\setcounter{enumi}{3}
  \item how to know if a point lies above or below another one?
  \item how to know if a point lies to the right of a line?
\end{enumerate}

Mathematically speaking, the answer to Question 4 is straightforward: a point $a$ lies below a point $q$ if $a_{y} < q_{y}$. For Question 5 the answer is$\ldots$

A point $q$ lies to the right of a line $\overrightarrow{l}_{ab}$ if the points $a$, $b$ and $q$ make a right turn, or if the signed area of the triangle $\Delta abq$ is negative.

Ok, I know what you are thinking$\ldots$ This is never going to end! How can I know if three points make a right turn?! Bare with me, I'm about to tell you. Three points $abq$ (organized in this order) make a right turn if:

\begin{equation}
  \label{eq:turn}
  \begin{split}
  2 \times \text{Area} (\Delta abq) = &
      \begin{vmatrix}
        a_{x} & a_{y} & 1 \\
        b_{x} & b_{y} & 1 \\
        q_{x} & q_{y} & 1
      \end{vmatrix} \\
  & = (b_{x}-a_{x})(q_{y}-a_{y})-(q_{x}-a_{x})(b_{y}-a_{y}) < 0
  \end{split}
\end{equation}

As a special case, if the expression above equals 0, the points $a$, $b$ and $q$ are collinear. We can now revisit Algorithm~\ref{inclusion-naive} and make it more tangible:

\begin{algorithm}
\caption{$PointInclusion(q, P)$ $-$ Revisited}
\textit{Input.} A point $q$ and a polygon $P$. \\
\textit{Output.} If the point is inside the polygon or not (boolean).
\begin{algorithmic}[1]
  \State{$N \gets 0$}
  \For{all edges $e_{ab}$ of $P$, where $a$ is the lower and $b$ is the upper point}
    \If{$a_{y} < q_{y}$ \textbf{and} $b_{y} > q_{y}$ \textbf{and} Equation~\ref{eq:turn} is satisfied}
      \State{$N = N + 1$}
    \EndIf{}
  \EndFor{}

  \If{$N$ is odd}
    \State{$q$ is inside $P$}
  \Else{}
    \State{$q$ is outside $P$}
  \EndIf{}
\end{algorithmic}\label{inclusion-revisited}
\end{algorithm}

\subsection{Implementation}\label{subsec:label}


\section{Degenerated cases}\label{subsec:point-degenerated}

\begin{itemize}
  \item \textbf{Case 1}: The line $r$ crosses the boundary of $P$ at a vertex $b$ there is part of two non-horizontal edges $ab$ and $bc$.
        \begin{itemize}
          \item The intersection with $r$ will be reported for both edges $ab$ and $bc$
          \item We should increment $N$ only for the edge(s) with the lower endpoint at $b$. There might be zero,  one or two of such case.
        \end{itemize}

  \item \textbf{Case 2}: $P$ has one or a few horizontal edges, and some of those fall on the line $r$.
        \begin{itemize}
          \item Ignore the horizontal edges
        \end{itemize}
\end{itemize}

\begin{algorithm}
\caption{$PointInclusion(q, P) - $ Revisited}
\textit{Input.} A point $q$ and a polygon $P$. \\
\textit{Output.} If the point is inside the polygon or not (boolean).
\begin{algorithmic}[1]
    \State{Draw a horizontal line $r$ from $q$ to its left}
    \For{all edges $l_{p}$ of $P$ that are non-horizontal}
      \If{$l_{p}$ intersects $r$ at a point other than its upper endpoint}
        \State{$N=N+1$}
      \EndIf{}
    \EndFor{}

    \If{$N$ is odd}
        \State{$q$ is inside $P$}
    \Else{}
        \State{$q$ is outside $P$}
    \EndIf{}
\end{algorithmic}\label{inclusion}
\end{algorithm}

\section{Special case $-$ Convex polygons}\label{subsec:point-convex}


Definition: a vertex which has its angle exactly equal $\pi$ is called ``degenerated''.

Tip: eliminate degenerated vertices.

A non-convex polygon can be subdivided into convex parts.

Spend time in pre-processing.

\begin{equation}
  \label{center-mass}
  C_{m} = \left ( \dfrac{v_{x} + u_{x} + w_{x}}{3}, \dfrac{v_{y} + u_{y} + w_{y}}{3} \right )
\end{equation}

\begin{equation}
  \label{tangent}
  \tan{\alpha} = \dfrac{q_{y} - z_{y}}{q_{x} - z_{x}}
\end{equation}



\begin{algorithm}
  \caption{$PointInclusionInConvexPolygon(q, P)$}
  \textit{Input.} A point $q$ and a convex polygon $P$. \\
  \textit{Output.} If the point is inside the polygon or not (boolean).
  \begin{algorithmic}[1]
    \State{Select three non-collinear vertices $a$, $b$ and $c$ of $P$.}
    \State{Calculate the points' center of mass $z$.}

    \For{each vertex $v$ of $P$}
      \State{Calculate the tangent of the angle $\alpha_{v}$ between an horizontal line $l_{0z}$ and $\overrightarrow{l}_{vz}$}
    \EndFor{}

    \State{Compute the angle $\beta$ between an horizontal line $l_{0z}$ and $\overrightarrow{l}_{qz}$}
    \State{The point $q$ is located at the wedge where $\tan{\alpha_{v_{1}}} < \tan{\beta} < \tan{\alpha_{v_{2}}}$}, being $v_{1}$ and $v_{2}$ two consecutive vertices of $P$.

    \If{$q$ is at the left of the edge $e$ of $P$ contained in this wedge}
      \State{$q$ is inside $P$}
    \Else{}
      \State{$q$ is outside $P$}
    \EndIf{}

  \end{algorithmic}
\end{algorithm}

Time complexity: $O(\log{n})$, where $n$ is the number of vertices of $P$. For the preprocessing it is required $O(n)$ time. Memory requirements: $O(n)$.


\chapter{Convex Hull}

\section{Convex Hull $-$ Naive}

\begin{algorithm}
\caption{$ConvexHullNaive(P)$}
\textit{Input.} A set $P$ of points in the plane. \\
\textit{Output.} A list $\mathcal{L}$ containing the vertices of $\mathcal{CH}(P)$ in clockwise order.
\begin{algorithmic}[1]
    \State{$E \gets 0$}
    \For{all ordered pairs $(p, q) \in P \times P$ with $p$ not equal to $q$}
        \State{\textit{valid} $\gets$ \textbf{true}}
        \For{all points in $r \in P$ not equal to $p$ or $q$}
            \If{$r$ lies to the left of the directed line from $p$ to $q$}
                \State{\textit{valid} $\gets$ \textbf{false}}
            \EndIf{}
        \EndFor{}
        \If{\textit{valid}}
            \State{Add the directed edge $\overrightarrow{pq}$ to $E$}
        \EndIf{}
    \EndFor{}
    \State{From the set $E$ of edges construct a list $\mathcal{L}$ of vertices of $\mathcal{CH}(P)$, sorted in clockwise order.}
\end{algorithmic}
\end{algorithm}

\subsubsection{Questions:}
\begin{enumerate}
  \item How to know if a point is to the left of a line segment?
  \item How to sort the vertices in clockwise order?
  \item How to know if two points are equal?
\end{enumerate}

\subsubsection{Objects, attributes and methods:}
\begin{enumerate}
  \item Points (Vertices)
        \begin{itemize}
          \item Attributes:
                \begin{itemize}
                  \item Id [int]
                  \item Coordinates x and y [Tuple (float, float)]
                \end{itemize}
        \end{itemize}
  \item Lines (Edges)
        \begin{itemize}
          \item Attributes:
                \begin{itemize}
                  \item Id [int]
                  \item Points v1 and v2 [Tuple (point, point)]
                \end{itemize}
          \item Methods:
               \begin{itemize}
                  \item For a given point, verify if the line is at its right
               \end{itemize}
        \end{itemize}
\end{enumerate}

\subsubsection{Data structures}
A list capable of sorting the points as they are included. The same point shouldn't be included twice.

% \section{Convex Hull - Fast}

% \begin{algorithm}[H]
% \caption{ConvexHull($P$)}
%     \begin{algorithmic}
%         \STATE{} \textit{Input.} A set $P$ of points in the plane.
%         \STATE{} \textit{Output.} A list $\mathcal{L}$ containing the vertices of $\mathcal{CH}(P)$ in clockwise order.

%         1. Sort the points by x-coordinate, resulting in a sequence $p_{1}, ..., p_{n}$ \\
%         2. Put the points $p_{1}$ and $p_{2}$ in a list $\mathcal{L}_{upper}$, with $p_{1}$ as the first point \\
%         3. \textbf{for} $i \gets 3$ to $n$ \\
%         4. \quad \textbf{do} Append $p_{i}$ to $\mathcal{L}_{upper}$ \\
%         5. \quad \quad \textbf{while} $\mathcal{L}_{upper}$ contains more than two points \textbf{and} the last three points in $\mathcal{L}_{upper}$ do not make a right turn \\
%         6. \quad \quad \quad \textbf{do} Delete the middle of the last three points from $\mathcal{L}_{upper}$ \\
%         7. Put the points $p_{n}$ and $p_{n-1}$ in a list $\mathcal{L}_{lower}$, with $p_{n}$ as the first point \\
%         8. \textbf{for} $i \gets (n - 2)$ \textbf{downto} 1 \\
%         9. \quad \textbf{do} Append $p_{i}$ to $\mathcal{L}_{lower}$ \\
%         10. \quad \quad \textbf{while} $\mathcal{L}_{lower}$ contains more than 2 points \textbf{and} the last three points in $\mathcal{L}_{lower}$ do not make a right turn \\
%         11. \quad \quad \quad \textbf{do} Delete the middle of the last three points from $\mathcal{L}_{lower}$ \\
%         12. Remove the first and the last point from $\mathcal{L}_{lower}$ to avoid duplication of the points where the upper and lower hull meet \\
%         13. Append $\mathcal{L}_{lower}$ to $\mathcal{L}_{upper}$, and call the resulting list $\mathcal{L}$ \\
%         14. \textbf{return} $\mathcal{L}$
%     \end{algorithmic}
% \end{algorithm}

\end{document}
