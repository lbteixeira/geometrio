\documentclass[12pt, a4paper]{article}
\usepackage[utf8]{inputenc}
\usepackage{algorithm}
\usepackage{amsmath}
\usepackage[noend]{algpseudocode}

\title{Computational Geometry Algorithms}
\author{Lucas Teixeira}
\date{\today}

\begin{document}

\maketitle


\section{Convex Hull}

\subsection{Convex Hull $-$ Naive}

\begin{algorithm}
\caption{$ConvexHullNaive(P)$}
\textit{Input.} A set $P$ of points in the plane. \\
\textit{Output.} A list $\mathcal{L}$ containing the vertices of $\mathcal{CH}(P)$ in clockwise order.
\begin{algorithmic}[1]
    \State{$E \gets 0$}
    \For{all ordered pairs $(p, q) \in P \times P$ with $p$ not equal to $q$}
        \State{\textit{valid} $\gets$ \textbf{true}}
        \For{all points in $r \in P$ not equal to $p$ or $q$}
            \If{$r$ lies to the left of the directed line from $p$ to $q$}
                \State{\textit{valid} $\gets$ \textbf{false}}
            \EndIf{}
        \EndFor{}
        \If{\textit{valid}}
            \State{Add the directed edge $\overrightarrow{pq}$ to $E$}
        \EndIf{}
    \EndFor{}
    \State{From the set $E$ of edges construct a list $\mathcal{L}$ of vertices of $\mathcal{CH}(P)$, sorted in clockwise order.}
\end{algorithmic}
\end{algorithm}

\subsubsection{Questions:}
\begin{enumerate}
  \item How to know if a point is to the left of a line segment?
  \item How to sort the vertices in clockwise order?
  \item How to know if two points are equal?
\end{enumerate}

\subsubsection{Objects, attributes and methods:}
\begin{enumerate}
  \item Points (Vertices)
        \begin{itemize}
          \item Attributes:
                \begin{itemize}
                  \item Id [int]
                  \item Coordinates x and y [Tuple (float, float)]
                \end{itemize}
        \end{itemize}
  \item Lines (Edges)
        \begin{itemize}
          \item Attributes:
                \begin{itemize}
                  \item Id [int]
                  \item Points v1 and v2 [Tuple (point, point)]
                \end{itemize}
          \item Methods:
               \begin{itemize}
                  \item For a given point, verify if the line is at its right
               \end{itemize}
        \end{itemize}
\end{enumerate}

\subsubsection{Data structures}
A list capable of sorting the points as they are included. The same point shouldn't be included twice.

% \subsection{Convex Hull - Fast}

% \begin{algorithm}[H]
% \caption{ConvexHull($P$)}
%     \begin{algorithmic}
%         \STATE{} \textit{Input.} A set $P$ of points in the plane.
%         \STATE{} \textit{Output.} A list $\mathcal{L}$ containing the vertices of $\mathcal{CH}(P)$ in clockwise order.

%         1. Sort the points by x-coordinate, resulting in a sequence $p_{1}, ..., p_{n}$ \\
%         2. Put the points $p_{1}$ and $p_{2}$ in a list $\mathcal{L}_{upper}$, with $p_{1}$ as the first point \\
%         3. \textbf{for} $i \gets 3$ to $n$ \\
%         4. \quad \textbf{do} Append $p_{i}$ to $\mathcal{L}_{upper}$ \\
%         5. \quad \quad \textbf{while} $\mathcal{L}_{upper}$ contains more than two points \textbf{and} the last three points in $\mathcal{L}_{upper}$ do not make a right turn \\
%         6. \quad \quad \quad \textbf{do} Delete the middle of the last three points from $\mathcal{L}_{upper}$ \\
%         7. Put the points $p_{n}$ and $p_{n-1}$ in a list $\mathcal{L}_{lower}$, with $p_{n}$ as the first point \\
%         8. \textbf{for} $i \gets (n - 2)$ \textbf{downto} 1 \\
%         9. \quad \textbf{do} Append $p_{i}$ to $\mathcal{L}_{lower}$ \\
%         10. \quad \quad \textbf{while} $\mathcal{L}_{lower}$ contains more than 2 points \textbf{and} the last three points in $\mathcal{L}_{lower}$ do not make a right turn \\
%         11. \quad \quad \quad \textbf{do} Delete the middle of the last three points from $\mathcal{L}_{lower}$ \\
%         12. Remove the first and the last point from $\mathcal{L}_{lower}$ to avoid duplication of the points where the upper and lower hull meet \\
%         13. Append $\mathcal{L}_{lower}$ to $\mathcal{L}_{upper}$, and call the resulting list $\mathcal{L}$ \\
%         14. \textbf{return} $\mathcal{L}$
%     \end{algorithmic}
% \end{algorithm}

\end{document}
